
%Preamble. You need all of this
%Document settings
\documentclass[letterpaper, 12pt, titlepage]{article}
%Package for physics notation
\usepackage{physics}
%Package to display chemical equaltions
\usepackage{mhchem}
%Package to use other math symbols
\usepackage{amsthm}
%Set the margins as 1 inch because the default is 2.5 and looks bad
\usepackage[margin=1in]{geometry}
%Package for plotting data and making diagrams
\usepackage{pgfplots}
%Package to better customize figure placement
\usepackage{float}
%Package to make subfloats (floats in floats in floats)
\usepackage{subfloat}
%Package with more arrows (for plotting on axis)
\usetikzlibrary{arrows.meta}
%Package for Bra-Ket notation
\usepackage{braket}
%Package for more math notation
\usepackage{amsmath}
%Even more math notation
\usepackage{amssymb}
%Change line spacing to 1.45
\linespread{1.45}

\begin{document}

\section{Theory}

\subsection{Hartree-Fock, Coupled Cluster, and M$\o$ller Plesset Methods}
\subsubsection{Hartree-Fock}
The Hartree-Fock (HF) or self-consistent field (SCF) method is a fundamental approach to solving electronic structure problems. The wave-function of our system, $\ket{\Psi}$, is represented by the determinant of a square matrix of one-electron orbital functions. For a two electron system, the wavefunction takes the form
\begin{align*}
\ket{\Psi}&= \frac{1}{\sqrt{2}} \,
\begin{tabular}{|c c|}
$\phi_1(x_1)$ &$\phi_1(x_2)$\\
$\phi_1(x_1)$&$\phi_2(x_2)$
\end{tabular}\\
\end{align*}
This notation can be simplified, and the wavefunction is written in shorthand using the ket symbol,
\begin{align*}
\ket{\Psi} &= \ket{\phi_i} \ket{\phi_j} ...\ket{\phi_n}\\
\ket{\Psi} &= \ket{\phi_i \phi_j ...\phi_n} \\
\end{align*}
Or, even more simply, as
\begin{align*}
\ket{\Psi} &= \ket{i} \ket{j} ...\ket{n}\\
\ket{\Psi} &= \ket{ij...n}
\end{align*}
The $\ket{i}, \ket{j} ...$ then correspond to occupied orbitals in the system. We will also introduce a notation corresponding to virtual orbitals;  $\ket{a}$, $\ket{b}...$. \\

The HF ground state energy is given by
\begin{align*}
E_{HF}&=\Bra{\ket{\Psi}}H\ket{\Psi}\\
&=\sum\limits_i \braket{i|h|i} + \frac{1}{2} \sum\limits_{i,j} \braket{ij||ij}
\end{align*}
Hartree-Fock is a fair starting point for systems which can be accurately described by a single determinant. Unfortunately, the results leave much to be desired. HF excludes energy contributions from electron correlation \cite{SO}. While these contributions are small in magnitude compared to the overall energy of the system, their absolute values are still quite large, and it is essential to include them when calculating energy differences, such as in thermochemical calculations. 
\subsubsection{Coupled Cluster}
Coupled cluster (CC) methods are a set of widely used methods which provide extremely accurate energies and geometries. They are also able to accurately determine a wide variety of difficult to predict properties such as NMR shifts and excited state properties. Our wave-function takes the form $\Psi=e^{\hat{T}} \ket{\Phi}$ where $\ket{\Phi}$ is the regular Hartree-Fock wavefunction.  $\hat{T}$ is the total excitation operator for the system,
\begin{align*}
\hat{T} &= \hat{T}_1 + \hat{T}_2 + \hat{T}_3...\\
\hat{T}_1 &= \sum t_i^a \{a^+i\}\\
\hat{T}_2 &= \frac{1}{(2!)^2} \sum t_{ij}^{ab} \{a^+ib^+j\}\\
\hat{T}_n &= \frac{1}{(n!)^2} \sum t_{ij...q}^{ab...p}\{a^+ib^+j...p^{+}q\}\\
\end{align*} 
The $a^+$ and $i$ refer to the creation and annihilation operators,  written in normal order, and the $t$ refer to t-amplitudes, which are parameters that must be determined. \\
\medskip
To evaluate $\Psi$, we use the Taylor expansion for $e^x$
\begin{align*}
e^{\hat{T}}&=\sum \frac{T^n}{n!}\\
e^{\hat{T}}&=(1+\hat{T} + \frac{\hat{T}^2}{2!} ...)\\
e^{\hat{T}}&=(1+\hat{T}_1 + \hat{T}_2 + \frac{\hat{T}^2_1}{2} + \hat{T}_1 \hat{T}_2 +...)
\end{align*}
The Schroedinger equation is written 
\begin{align*}
He^{\hat{T}} \ket{\Phi} &=  e^{\hat{T}} \ket{\Phi} E \\
e^{-\hat{T}}He^{\hat{T}} \ket{\Phi} &=  e^{-\hat{T}}e^{\hat{T}} \ket{\Phi} E\\
\mathbf{\bar{H}} \ket{\Phi} &=  \ket{\Phi} E 
\end{align*}
We now obtain the coupled cluster equations, which are used to solve for T amplitudes
\begin{align*}
\bra{\Phi}e^{-\hat{T}}He^{\hat{T}}\ket{\Phi} =& E \braket{\Phi | \Phi} = E\\
\bra{\Phi_i^{a}}e^{-\hat{T}}He^{\hat{T}}\ket{\Phi} =& E\braket{\Phi_i^{a}|\Phi} = 0\\
\bra{\Phi_{ij}^{ab}}e^{-\hat{T}}He^{\hat{T}}\ket{\Phi} =& E \braket{\Phi_{ij}^{ab} | \Phi} = 0\\
&\vdots
\end{align*}
Once the CC equations are solved, the ground state energy is readily available.\\

If excited state properties are desired, it is possible to diagonalize $\mathbf{\bar{H}}$ over the states of interest (e.g. singles and doubles). This extension is called equation of motion coupled cluster (EOM-CC) and is an extremely useful tool in calculating excited state properties. Full CC includes contributions from all possible electronic configurations and it can be shown that the  CC methods converge to the full configuration-interaction (CI) solution. The most common implementations of CC methods are CC singles and doubles (CCSD) and CC singles, doubles, triples (CCSDT), which, as the names imply, truncate with the $\hat{T}_2$ and $\hat{T}_3$ excitation operators, respectively.\\

\subsubsection{MP}
M$\o$ller-Plesset perturbation theory (MP) is a simpler method in which a starting Hartree-Fock calculation is perturbatively corrected to include electron correlation energies. The Schroedinger equation used in MP methods is
\begin{align*}
\hat{H}\Psi&=(\hat{H}^0 +\lambda \hat{V}) \Psi=E\Psi &\lambda \in \mathbb{R}\\
\end{align*} 
Where $\hat{V}$ is the correction to the unperturbed Hamiltonian, $\hat{H}^0$, and $\lambda$ is an arbitrary parameter which controls the level of perturbation. The n-th order terms are determined by expanding $\Psi$ and E in terms of powers of $\lambda$.
\begin{align*}
(\hat{H}^0 + \hat{V}) \sum\limits_i^n \lambda^i \Psi^{(i)} &= \sum\limits_j^n \lambda^j E^{(j)} \sum\limits_k^n \lambda^k \Psi^{(k)}  \\
\end{align*}
The zeroth order (uncorrected) energy is the energy of the unperturbed system, as expected. The first order correction to the energy ($E^{(1)}$) is given by
\begin{align*}
\hat{H}^0 \ket{\Psi^{(1)}} + \hat{V}\ket{\Psi^{(0)}}&=E^{(0)}\ket{\Psi^{(1)}} + E^{(1)}\ket{\Psi^{(0)}}\\
\bra{\Psi^{(0)}}\hat{H}^0 \ket{\Psi^{(1)}} + \bra{\Psi^{(0)}}\hat{V}\ket{\Psi^{(0)}} &=\bra{\Psi^{(0)}}E^{(0)}\ket{\Psi^{(1)}} +\bra{\Psi^{(0)}} E^{(1)}\ket{\Psi^{(0)}}\\
\bra{\Psi^{(0)}}\hat{V}\ket{\Psi^{(0)}} &= E^{(1)} \braket{\Psi^{(0)} | \Psi^{(0)}}\\
E^{(1)} &= \bra{\Psi^{(0)}}\hat{V}\ket{\Psi^{(0)}} 
\end{align*}
Which is the expectation value of the correction.  The most widely used version of MP theory is MP2, which truncates after the n=2 terms. The MP2 correction to the Hartree-Fock state is given by 
\begin{align*}
E^{(2)} = \sum\limits_{i,j}^{N_{vert}} \sum\limits_{a,b}^{N_{occ}} \frac{|\braket{ab||ij}|^2}{\varepsilon_a + \varepsilon_b -\varepsilon_i-\varepsilon_j}
\end{align*} 
Since MP2 is just a Hartree-Fock calculation with a perturbative correction, it is extremely cheap and gives better results than HF. Since MP2 is a non-variational method, it may give energies below the ground state \cite{SO}. CC methods may make use of MP theory to quickly calculate contributions from higher-ordered excitations. An example of this is CCSD(T) which uses pertubation theory to calculate the contributions from triple excitations. MP2, CCSD, CCSD(T) and CCSDT scale as $N^4$, $N^6$, $N^7$, and $N^8$, respectively, while Hartree-Fock scales as $N^4$ \cite{nooijen}.
\end{document}